\documentclass[a4paper]{article}

\usepackage[a4paper,  margin=1.0in]{geometry}

\usepackage{graphicx}
\usepackage{float}
\usepackage{hyperref}




\usepackage{polski}
\usepackage[utf8]{inputenc}
\begin{document}


\title{ Laboratorium Rozpoznawania Obrazów – Ćwiczenie \#3 \& \#4 Rozpoznawanie cyfr pisanych ręcznie }


\author{Michał Sypetkowski}
\maketitle


\section{Dane i ogólne uwagi}
Eksperymenty przeprowadzane są na zbiorze danych MNIST. 
Mamy 60k przykładów trenujących i 10k do testowania.

Zbiór trenujący - pliki:
\begin{verbatim}
train-images-idx3-ubyte
train-labels-idx1-ubyte
\end{verbatim}

Zbiór testujący:
\begin{verbatim}
t10k-images-idx3-ubyte
t10k-labels-idx1-ubyte
\end{verbatim}

Skrypt \texttt{genData.m} zmniejsza liczbę wymiarów za pomocą PCA
i zapisuje dane używane później do eksperymentóœ w plikach:
\begin{verbatim}
train trainl test testl
\end{verbatim}



\end{document}
