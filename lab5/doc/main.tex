\documentclass[a4paper]{article}

\usepackage[a4paper,  margin=0.4in]{geometry}

\usepackage{graphicx}
\usepackage{float}
\usepackage{hyperref}
\usepackage{pgfplots}
\usepackage{multicol}
\usepackage{multirow}




% \usepackage{polski}
\usepackage[utf8]{inputenc}

\begin{document}


\title{Pattern Recognition Laboratory – Assignment \#7
Recognition quality enhancement (handwritten digits)}

\author{Michał Sypetkowski}
\maketitle

\section{General information}
Simple thresholding experiment is implemented in \texttt{mainScript2.m}.

\texttt{loadCNNOutputs.m} has additional parameter \texttt{probabilities}.
When set to true, it returns matrix 10000x10 -- mean probability vectors.

\section{Simple probability thresholding}
First, I calculate probability vector by taking mean vectors
(mean from vectors of each of 7 classifiers).
The answer is argmax of probability vector.
Then by simply thresholding max probability with 0.79 (constant selected experimentally) for rejection, I achieved
better results (larger values objective function) than unanimity voting.
The results are shown in table \ref{table:res1}.


\begin{table}[H]
    \caption{Results comparison
    \label{table:res1}.
    }
\begin{center}
    \begin{tabular}{| p{4cm} | l | l | l | l | l | l | l | l | l | l | l | l |}

    \hline
    \multirow{2}{*} {empty} & 
        \multicolumn{4}{|l|} {Unanimity Voting} &
        \multicolumn{4}{|l|} {Simple thresholding}
        \\ \cline{2-9}
        & OK & Error & reject & objective
        & OK & Error & reject & objective
        \\
    \hline

    Coefficients and objective function for validation set
    & 97.37 & 0.79 & 1.84 & 94.21 
    & 97.08 &  0.58 &  2.34 & 94.76
    \\ \cline{2-9}

    Coefficients and objective function for test set
        & 97.72 & 0.52 & 1.76 & 95.64
        & 97.33 & 0.36 & 2.31 & 95.89
    \\ \cline{2-9}



    \hline


    \end{tabular}
\end{center}
\end{table}


To sum up, even simple probability thresholding gives better results than any of 3 types of voting for our objective function.
\end{document}
