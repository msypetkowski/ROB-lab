\documentclass[a4paper]{article}

\usepackage[a4paper,  margin=1.0in]{geometry}

\usepackage{graphicx}
\usepackage{float}
\usepackage{hyperref}




\usepackage{polski}
\usepackage[utf8]{inputenc}
\begin{document}


\title{ Laboratorium Rozpoznawania Obrazów – Ćwiczenie \#3 \& \#4 Rozpoznawanie cyfr pisanych ręcznie }


\author{Michał Sypetkowski}
\maketitle



% TODO ma być w doc:
% jakość każdego z 45 klasyfikatorów
% od 7 jest 9 ekspertów i 36 nieuków

% 2 część do zrobienia jest ciekawsza
% ograniczyć rozległość zespołów jak badamy każda cyfra z każdą

% jak dokonamy podziału na grupy to będzie głosowało 12 klasyfikatorów a nie 45
% w różnych grupach uzyskamy różną jakość (zależy jak dobże się cyferki oddzielają)

% ostatnia rzecz do zrobienia:
% klasyfikator który powie że to jest cyferka z danej grupy (np. A lub B lub C)
% one vs one?

% A 3 5 8
% B 4 7 9
% B 0 1 2 6
% 12 ?

% Podział musi być naprawde dobry żeby wynik był lepszy
% a to dużo trzebaby siedizeć, ale szkoda czasu

% confustion passes - check

\section{Dane i ogólne uwagi}
Eksperymenty przeprowadzane są na zbiorze danych MNIST. 
Mamy 60k przykładów trenujących i 10k do testowania.

Zbiór trenujący - pliki:
\begin{verbatim}
train-images-idx3-ubyte
train-labels-idx1-ubyte
\end{verbatim}

Zbiór testujący:
\begin{verbatim}
t10k-images-idx3-ubyte
t10k-labels-idx1-ubyte
\end{verbatim}

Skrypt \texttt{genData.m} zmniejsza liczbę wymiarów za pomocą PCA
i zapisuje dane używane później do eksperymentów w plikach:
\begin{verbatim}
train trainl test testl
\end{verbatim}

\section{Testy algorytmu dla wielowymiarowych danych.}

Zależność ostatecznej precyzji w zależności od ilości wymiarów danych jest przedstawianoa na wykresie \ref{TODO}.
Eksperyment z wymiarem 784 odpowiada braku redukcji algorytmem PCA. % TODO: wiadomo


\section{Ulepszenie rozwiązania}
% pomysł: subuj prawdopodobieńśtwa w głosowaniu a nie decyzje


\end{document}
