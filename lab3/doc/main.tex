\documentclass[a4paper]{article}

\usepackage[a4paper,  margin=1.0in]{geometry}

\usepackage{graphicx}
\usepackage{float}
\usepackage{hyperref}




\usepackage{polski}
\usepackage[utf8]{inputenc}
\begin{document}


\title{ Laboratorium Rozpoznawania Obrazów – Ćwiczenie \#3 \& \#4 Rozpoznawanie cyfr pisanych ręcznie }


\author{Michał Sypetkowski}
\maketitle



% TODO ma być w doc:
% jakość każdego z 45 klasyfikatorów
% od 7 jest 9 ekspertów i 36 nieuków

% 2 część do zrobienia jest ciekawsza
% ograniczyć rozległość zespołów jak badamy każda cyfra z każdą

% jak dokonamy podziału na grupy to będzie głosowało 12 klasyfikatorów a nie 45
% w różnych grupach uzyskamy różną jakość (zależy jak dobże się cyferki oddzielają)

% ostatnia rzecz do zrobienia:
% klasyfikator który powie że to jest cyferka z danej grupy (np. A lub B lub C)
% one vs one?

% A 3 5 8
% B 4 7 9
% B 0 1 2 6
% 12 ?

% Podział musi być naprawde dobry żeby wynik był lepszy
% a to dużo trzebaby siedizeć, ale szkoda czasu

% confustion passes - check

\section{Dane i ogólne uwagi}
Eksperymenty przeprowadzane są na zbiorze danych MNIST. 
Mamy 60k przykładów trenujących i 10k do testowania.

Zbiór trenujący - pliki:
\begin{verbatim}
train-images-idx3-ubyte
train-labels-idx1-ubyte
\end{verbatim}

Zbiór testujący:
\begin{verbatim}
t10k-images-idx3-ubyte
t10k-labels-idx1-ubyte
\end{verbatim}

Skrypt \texttt{genData.m} zmniejsza liczbę wymiarów za pomocą PCA
i zapisuje dane używane później do eksperymentów w plikach:
\begin{verbatim}
train trainl test testl
\end{verbatim}

Eksperymenty opisane w sekcji \ref{canonical} są zaimplementowane w pliku \texttt{main.m},
a te opisane w sekcji \ref{improved} -- w pliku \texttt{main2.m}.

\section{Rozwiązanie "kanoniczne"}
\label{canonical}

W rozwiązaniu kanonicznym trenowane jest 45 klasyfikatorów liniowych
i przeprowadzane jest głosowanie typu OVO.
Trenowanie pojedyńczego klasyfikatora jest zaimplementowane w pliku \texttt{perceptron.m}.
Każdy klasyfikator (spośród 45) był trenowany po 10 razy aby wybrać ten najlepszy.

\subsection{Testy algorytmu dla wielowymiarowych danych.}
Wyniki klasyfikatora w zależności od liczby atrybutów (redukcja PCA) przedstawione są w tebeli \ref{table:attrCount},
a wyniki na zbiorze trenującym w tabeli \ref{table:attrCountTrain}.

W razie potrzeby wyciągnięcia bardziej szczegółowych danych,
eksperymenty można powtórzyć (ustawione zostało stałe ziarno losowe).

\begin{table}[H]
    \caption{Wyniki klasyfikatora "kanonicznego" dla różnej ilości atrybutów - na zbiorze testowym
    \label{table:attrCount}
    }
\begin{center}
    \begin{tabular}{| l | l | l | l |}
    \hline
        Ilość atrybutów & precyzja & błąd & wsp. odrzuconych \\
    \hline
        2  & 0.374 & 0.447 & 0.179 \\
        10 & 0.797 & 0.151 & 0.052 \\
        20 & 0.893 & 0.073 & 0.033 \\
        40 & 0.914 & 0.056 & 0.030 \\
        80 & 0.923 & 0.048 & 0.029 \\
        160& 0.924 & 0.049 & 0.028 \\
        784& 0.919 & 0.046 & 0.035 \\
        784 - bez PCA & 0.925 & 0.051 & 0.025 \\
    \hline
    \end{tabular}
\end{center}
\end{table}

\begin{table}[H]
    \caption{Wyniki klasyfikatora "kanonicznego" dla różnej ilości atrybutów - na zbiorze trenującym
    \label{table:attrCountTrain}
    }
\begin{center}
    \begin{tabular}{| l | l | l | l |}
    \hline
        Ilość atrybutów & precyzja & błąd & wsp. odrzuconych \\
    \hline
        2  & 0.372 & 0.451 & 0.176 \\
        10 & 0.793 & 0.152 & 0.055 \\
        20 & 0.888 & 0.079 & 0.033 \\
        40 & 0.912 & 0.057 & 0.030 \\
        80 & 0.925 & 0.048 & 0.027 \\
        160& 0.930 & 0.044 & 0.026 \\
        784& 0.935 & 0.038 & 0.028 \\
        784 - bez PCA & 0.928 & 0.046 & 0.027 \\
    \hline
    \end{tabular}
\end{center}
\end{table}

Wynik dla 160 wymiarowych danych niewiele różni się od tego dla 80 wymiarowych.
Jako rozwiązanie "kanoniczne" będzie przyjmowany klasyfikator działający na 80 atrybutach.

\subsection{Szczegółowe wyniki dla klasyfikatora działającego na 80 atrybutach}


\begin{table}[H]
    \caption{Macierz pomyłek dla klasyfikatora "kanonicznego"
    \label{table:confusion}
    }
\begin{center}
    \begin{tabular}{| l | l | l | l | l | l | l | l | l | l | l | l | l |}
    \hline
        - & 0 & 1 & 2 & 3 & 4 & 5 & 6 & 7 & 8 & 9 & reject\\
    \hline
        0 &  950  &   0   &  0  &   1  &   0  &   5  &   5 &    0  &   0 &    1   & 18 \\
        1 &  950  &1096   &  5  &   1  &   0  &   1  &   2 &    1  &  16 &    0   & 13 \\
        2 &  951  &   2   &958  &   6  &   5  &   1  &   6 &    5  &  19 &    0   & 29 \\
        3 &  950  &   1   &  5  & 928  &   1  &  18  &   1 &    8  &  10 &    4   & 34 \\
        4 &  951  &   0   &  5  &   1  & 915  &   0  &   6 &    2  &   5 &   20   & 27 \\
        5 &  957  &   1   &  0  &  36  &   3  & 781  &   9 &    1  &  17 &    3   & 34 \\
        6 &  955  &   2   &  9  &   1  &   2  &  18  & 899 &    1  &   1 &    0   & 20 \\
        7 &  950  &   3   & 19  &   6  &   2  &   0  &   0 &  940  &   1 &   21   & 36 \\
        8 &  953  &   0   &  3  &  27  &   3  &  20  &   3 &    3  & 862 &    4   & 46 \\
        9 &  953  &   5   &  1  &   6  &  31  &   4  &   0 &   21  &   5 &  903   & 30 \\
    \hline
    \end{tabular}
\end{center}
\end{table}


\begin{table}[H]
    \caption{Wyniki pojedyńczych klasyfikatorów
    \label{table:indiv}
    }
\begin{center}
    \begin{tabular}{| l | l | l | l | l | l | l | l | l | l | l | l | l |}
    \hline
        - & 0 & 1 & 2 & 3 & 4 & 5 & 6 & 7 & 8 & 9 \\
    \hline
 0 &  0.000 & 0.999 & 0.991 & 0.993 & 0.997 & 0.990 & 0.990 & 0.995 & 0.993 & 0.994 \\
 1 &  0.000 & 0.000 & 0.993 & 0.995 & 1.000 & 0.993 & 0.995 & 0.992 & 0.985 & 0.995 \\
 2 &  0.000 & 0.000 & 0.000 & 0.977 & 0.982 & 0.979 & 0.983 & 0.978 & 0.978 & 0.981 \\
 3 &  0.000 & 0.000 & 0.000 & 0.000 & 0.995 & 0.958 & 0.992 & 0.980 & 0.961 & 0.981 \\
 4 &  0.000 & 0.000 & 0.000 & 0.000 & 0.000 & 0.988 & 0.989 & 0.989 & 0.991 & 0.965 \\
 5 &  0.000 & 0.000 & 0.000 & 0.000 & 0.000 & 0.000 & 0.972 & 0.993 & 0.956 & 0.985 \\
 6 &  0.000 & 0.000 & 0.000 & 0.000 & 0.000 & 0.000 & 0.000 & 0.993 & 0.990 & 0.995 \\
 7 &  0.000 & 0.000 & 0.000 & 0.000 & 0.000 & 0.000 & 0.000 & 0.000 & 0.991 & 0.953 \\
 8 &  0.000 & 0.000 & 0.000 & 0.000 & 0.000 & 0.000 & 0.000 & 0.000 & 0.000 & 0.981 \\
 9 &  0.000 & 0.000 & 0.000 & 0.000 & 0.000 & 0.000 & 0.000 & 0.000 & 0.000 & 0.000 \\
    \hline
    \end{tabular}
\end{center}
\end{table}


\section{Ulepszenie rozwiązania "kanonicznego"}
\label{improved}
% pomysł: subuj prawdopodobieńśtwa w głosowaniu a nie decyzje


\end{document}
